%%% Please fill in basic information on your thesis, which will be automatically
%%% inserted at the right places.

% Type of your thesis:
%	"bc" for Bachelor's
%	"mgr" for Master's
%	"phd" for PhD
%	"rig" for rigorosum
\def\ThesisType{bc}

% Language of your study programme:
%	"cs" for Czech
%	"en" for English
\def\StudyLanguage{cs}

% Thesis title in English (exactly as in the official assignment)
% (Note: \xxx is a "ToDo label" which makes the unfilled visible. Remove it.)
\def\ThesisTitle{User-oriented application for multimodal public transit route search in Prague}

% Author of the thesis (you)
\def\ThesisAuthor{Matěj Šubrt}

% Year when the thesis is submitted
\def\YearSubmitted{2024}

% Name of the department or institute, where the work was officially assigned
% (according to the Organizational Structure of MFF UK in English,
% see https://www.mff.cuni.cz/en/faculty/organizational-structure,
% or a full name of a department outside MFF)
\def\Department{Department of Software Engineering}

% Is it a department (katedra), or an institute (ústav)?
\def\DeptType{Department}

% Thesis supervisor: name, surname and titles
\def\Supervisor{Martin Nečaský}

% Supervisor's department (again according to Organizational structure of MFF)
\def\SupervisorsDepartment{Department of Software Engineering}

% Study programme (does not apply to rigorosum theses)
\def\StudyProgramme{\xxx{Programming and software development}}

% An optional dedication: you can thank whomever you wish (your supervisor,
% consultant, who provided you with tea and pizza, etc.)
\def\Dedication{%
I would like to thank my supervisor doc. Mgr. Martin Nečaský, Ph.D. for his help and advice towards this thesis. I would also like to thank my family for supporting me throughout my studies and the work on this thesis.
}

% Abstract (recommended length around 80-200 words; this is not a copy of your thesis assignment!)
\def\Abstract{%
There are many solutions for searching for a public transit connection both worldwide and in the Czech republic. They are however targeted at the widest possible audience, meaning they are simple to use and targeted at a broad area, but lack more advanced configuration options and typically are not optimized for use in single specific cities. This universal design overlooks the fact that most users primarily seek guidance within a particular city along routes they already partially know, and thus can provide valuable input towards optimizing their travels and commutes in the city. The application developed as a part of this thesis addresses these issues and aims to provide customizable and optimized connection searching to people using the Prague public transport network. The application also integrates bikesharing into the connection search, allowing for a more seamless experience for anyone traveling through the city.
}

% 3 to 5 keywords (recommended) separated by \sep
% Keywords are useful for indexing and searching for the theses by topic.
\def\ThesisKeywords{%
route search \sep open data \sep public transit \sep Prague \sep multimodal transit
}

% If any of your metadata strings contains TeX macros, you need to provide
% a plain-text version for use in XMP metadata embedded in the output PDF file.
% If you are not sure, check the generated thesis.xmpdata file.
\def\ThesisAuthorXMP{\ThesisAuthor}
\def\ThesisTitleXMP{\ThesisTitle}
\def\ThesisKeywordsXMP{\ThesisKeywords}
\def\AbstractXMP{\Abstract}

% If your abstracts are long and do not fit in the infopage, you can make the
% fonts a bit smaller by this setting. (Also, you should try to compress your abstract more.)
\def\InfoPageFont{}
%\def\InfoPageFont{\small}  % uncomment to decrease font size

% If you are studing in a Czech programme, you also need to provide metadata in Czech:
% (in English programmes, this is not used anywhere)

\def\ThesisTitleCS{Uživatelsky orientovaná aplikace pro vyhledávání multimodálních spojení v pražské hromadné dopravě}
\def\DepartmentCS{Katedra softwarového inženýrství}
\def\DeptTypeCS{Katedra}
\def\SupervisorsDepartmentCS{Katedra softwarového inženýrství}
\def\StudyProgrammeCS{Programování a vývoj software}

\def\ThesisKeywordsCS{%
vyhledávání tras \sep otevřená data \sep MHD \sep Praha \sep multimodální doprava
}

\def\AbstractCS{%
\xxx{Abstrakt práce přeložte také do češtiny.}
}
